%%% LaTeX Template
%%% This template is made for project reports
%%% You may adjust it to your own needs/purposes
%%%
%%% Copyright: http://www.howtotex.com/
%%% Date: March 2011

%%% Preamble
\documentclass[paper=a4, fontsize=11pt]{scrartcl} % Article class of KOMA-script with 11pt font and a4 format
\usepackage[T1]{fontenc}
\usepackage{fourier}

\usepackage[english]{babel}                             % English language/hyphenation
\usepackage[protrusion=true,expansion=true]{microtype}        % Better typography
\usepackage{amsmath,amsfonts,amsthm}                    % Math packages
\usepackage[pdftex]{graphicx}                           % Enable pdflatex
\usepackage{url}
\usepackage{parskip}


%%% Custom sectioning (sectsty package)
\usepackage{sectsty}                        % Custom sectioning (see below)
\allsectionsfont{\centering \normalfont\scshape}  % Change font of al section commands


%%% Custom headers/footers (fancyhdr package)
\usepackage{fancyhdr}
\pagestyle{fancyplain}
\fancyhead{}                            % No page header
\fancyfoot[C]{}                         % Empty
\fancyfoot[R]{\thepage}                 % Pagenumbering
\renewcommand{\headrulewidth}{0pt}      % Remove header underlines
\renewcommand{\footrulewidth}{0pt}        % Remove footer underlines
\setlength{\headheight}{13.6pt}

%%% Theorems, definitions, shorthand
\newcommand{\script}{\mathcal}
\theoremstyle{definition}
\newtheorem*{defn}{Definition}

%%% Equation and float numbering
\numberwithin{equation}{section}    % Equationnumbering: section.eq#
\numberwithin{figure}{section}      % Figurenumbering: section.fig#
\numberwithin{table}{section}       % Tablenumbering: section.tab#


%%% Maketitle metadata
\newcommand{\horrule}[1]{\rule{\linewidth}{#1}}   % Horizontal rule

\title{
    %\vspace{-1in}
    \usefont{OT1}{bch}{b}{n}
    %\normalfont \normalsize \textsc{School of random department names} \\ [25pt]
    \horrule{0.5pt} \\[0.4cm]
    \huge Topology -- The Order, Product and Subspace Topologies \\
    \horrule{2pt} \\[0.5cm]
}
\author{
    \normalfont                 \normalsize
        Thimoteus\\[-3pt]    \normalsize
        \today
}
\date{}


%%% Begin document
\begin{document}
\maketitle

\begin{defn}[Order topology]
  Generated by the basis
  \begin{align*}
    \script B &=    \{ x \in X \mid a < x < b \} \\
              &\cup \{ x \in X \mid a < x \le \max(X) \} \\
              &\cup \{ x \in X \mid \min(X) \le x < b \} \\
  \end{align*}
  if $\max(X), \min(X)$ are well-defined.
\end{defn}

\textbf{Note:} Open rays form a subbasis for the order topology.

\hrulefill

\begin{defn}[Product topology]
  If $X, Y$ are topologies, the order topology on $X \times Y$ is defined to be
  generated by the basis
  \[ \script B = \{ U \times V \mid U \in \script T_X, V \in \script T_Y \} \]
\end{defn}

\textbf{Theorem:} If $\script B$ is a basis for $X$ and $\script C$ for $Y$, then
the following is a basis for $X \times Y$:
\[ \{ B \times C \mid B \in \script B, C \in \script C \} \]

\textbf{Theorem:} The following is a subbasis for the product topology on $X \times $Y:
\[ S = \left\{ \pi_1^{-1}(U) \mid U \in \script T_X \right\} \cup \left\{ \pi_2^{-1}(V) \mid V \in \script T_Y \right\} \]

\hrulefill

\begin{defn}[Subspace topology]
  If $X$ is a topology and $Y \subseteq X$ then the following is the subspace
  topology on $Y$:
  \[ \{ Y \cap U \mid U \in \script T_X \} \]
\end{defn}

\textbf{Lemma:} If $\script B$ is a basis for the topology on $X$ then the
following is a basis for the subspace topology on $Y$:
\[ \{ B \cap Y \mid B \in \script B \} \]

\textbf{Lemma:} Let $Y$ be a subspace of $X$. If $U$ is open in $Y$ and $Y$ is
open in $X$ then $U$ is open in $X$.

\hrulefill

\textbf{Theorem:} If $A$ is a subspace of $X$ and $B$ is a subspace of $Y$ then
the product topology on $A \times B$ is the same as the subspace topology
$A \times B$ inherits from $X \times Y$.

\begin{defn}[Convex set]
  Let $Y$ be a subset of a totally ordered $X$. We say $Y$ is \textbf{convex}
  if for each $a, b \in Y$ the set $\{ x \in X \mid a < x < b \}$ is a subset
  of $Y$.
\end{defn}

\textbf{Theorem:} Let $X$ have an order topology and $Y$ a convex subset of $X$.
Then the order topology of $Y$ is the same as the topology it inherits from $X$
as a subspace.

%%% End document
\end{document}
